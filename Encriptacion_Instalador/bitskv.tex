\documentclass[11pt]{article}
\usepackage{listings}
\usepackage[T1]{fontenc}
\usepackage[utf8]{inputenc}
\usepackage{amsmath}
\usepackage{systeme}
\usepackage{array}
\usepackage[table, svgnames]{xcolor}

\title{\textbf{Bit SKV}}
\author{Bit SRL\\
		Equipo del SLTA-verse}
\date{}

\begin{document}
\maketitle

\section{Descripción}
SKV es un método de rápida autenticación de claves de instalación utilizando multiplicación de matrices, utilizando 4 matrices (el formato de encodeo de las matrices todavía está por definirse).

El cliente de SLTA tendrá (en algún formato no directamente visible) 4 matrices, $ A_{3}, B_{3}, C_{2}, D_{2} $ donde $ A \times B = M1$ y $ C \times D = M2 $, donde $M1_{3} y M2_{2}$ se describen en las siguientes secciones.

\section{M2}
M2 es una matriz de tamaño $2\times2$ con los siguientes valores:

\[
\begin{bmatrix}
S & T \\
L & A
\end{bmatrix}
\xrightarrow{\text{letras a números}}
\begin{bmatrix}
19 & 20 \\
12 & 1
\end{bmatrix}
\]
Para llegar a este resultado, necesitaremos 2 matrices $C_{2}, D_{2} / C \times D = M2$.

Por ejemplo, despejando del siguiente modo podemos llegar a valores ejemplo de C y D.
\newpage
\subsection{M2: C y D ejemplares}
Empezamos declarando la ecuación
\[
C \times D = M2
\]
Remplazamos M2 por su valor, y C,D por matrices con variables

\[
\begin{bmatrix}
a & b \\
c & d
\end{bmatrix}
\times
\begin{bmatrix}
j & k \\
l & m
\end{bmatrix}
=
\begin{bmatrix}
19 & 20 \\
12 & 1
\end{bmatrix}
\]
Damos a la matriz C valores para poder despejar la matriz D

\[
\begin{bmatrix}
5 & 12 \\
8 & 0
\end{bmatrix}
\times
\begin{bmatrix}
j & k \\
l & m
\end{bmatrix}
=
\begin{bmatrix}
19 & 20 \\
12 & 1
\end{bmatrix}
\]
Definimos las ecuaciones que se están realizando dentro de esta multiplicación (o sea, los dot product entre las filas de C y las columnas de D)
\begin{align*}
M2_{11}&=5.j + 12.l&=19\\
M2_{12}&=5.k + 12.m&=20\\
M2_{21}&=8.j + 0.l&=12\\
M2_{22}&=8.j + 0.m&=1\\
\end{align*}
\subsection{M2: primer sistema de ecuaciones}
Tomamos las ecuaciones de la columna 1 en un sistema de ecuaciones (ya que tienen las mismas variables)
\begin{equation*}
\systeme{
5j+12l = 19,
8j = 12
}
\end{equation*}
\begin{enumerate}
\item $8j = 12 \rightarrow \boxed{j=\frac{3}{2}}$
\item $5.\frac{3}{2} + 12.l=19 \rightarrow 12l=19 - \frac{15}{2} = \frac{23}{2} \rightarrow \boxed{l = \frac{23}{24}}$
\end{enumerate}
\newpage
\subsection{M2: segundo sistema de ecuaciones}
Tomamos las ecuaciones de la columna 2 en otro sistema de ecuaciones
\begin{equation*}
\systeme{
5k+12m = 20,
8k = 1
}
\end{equation*}
\begin{enumerate}
\item $8k = 1 \rightarrow \boxed{k=\frac{1}{8}}$
\item $5.\frac{1}{8} + 12m = 20 \rightarrow 12m = \frac{160-5}{8} \rightarrow \frac{96m}{8} = \frac{155}{8} \rightarrow \boxed{m = \frac{155}{96}}$
\end{enumerate}
\subsection{M2: Resultado}
Tras estos sistemas de ecuaciones, tenemos
\[
C = \begin{bmatrix}
5 & 12\\
8 & 0
\end{bmatrix}, D = \begin{bmatrix}
\frac{3}{2} & \frac{1}{8} \\
\frac{23}{24} & \frac{155}{96}
\end{bmatrix}
\]
Verificamos que $ C \times D $ sea igual a M2
\begin{align*}
(C \times D)_{11}&=5.\frac{3}{2} + 12.\frac{23}{24}&=\frac{456}{24}&=19\\
(C \times D)_{12}&=\frac{5}{8} + 12.\frac{155}{96}&=\frac{1920}{96}&=20\\
(C \times D)_{21}&=8.\frac{3}{2} + 0.\frac{23}{24}&&=12\\
(C \times D)_{22}&=8.\frac{1}{8} + 0.\frac{155}{96}&&=1\\
\end{align*}
\[
=
\begin{bmatrix}
19 & 20\\
12 & 1
\end{bmatrix}
\]
Lo cual significa que $\{C, D\}$ es parte de una clave válida para el sistema de Bit SKV.
\pagebreak
\section{M1}
M1 es una matriz de tamaño $3\times3$ con los siguientes
valores:
\[
\begin{bmatrix}
B & S & S \\
I & R & K \\
T & L & V
\end{bmatrix}
\xrightarrow{\text{letras a números}}
\begin{bmatrix}
2 & 19 & 19 \\
9 & 18 & 11 \\
20 & 11 & 22
\end{bmatrix}
\]
Para llegar a este resultado, necesitaremos 2 matrices
$A_{3}, B_{3} / A \times B = M1$.

Por ejemplo, despejando del siguiente modo podemos llegar a valores ejemplo de A y B.

\subsection{M1: A y B ejemplares}
Empezamos declarando la ecuación
\[
A \times B = M1
\]
Remplazamos M1 por su valor, y A,B por Matrices con variables

\[
\begin{bmatrix}
a & b & c \\
d & e & f \\
g & h & i
\end{bmatrix}
\times
\begin{bmatrix}
j & k & l \\
m & n & o \\
p & q & r
\end{bmatrix}
=
\begin{bmatrix}
2 & 19 & 19 \\
9 & 18 & 11 \\
20 & 11 & 22
\end{bmatrix}
\]
Damos a la matriz A valores para poder despejar la matriz B

\[
\begin{bmatrix}
1 & 3 & 0 \\
5 & 8 & 12 \\
9 & 4 & 10
\end{bmatrix}
\times
\begin{bmatrix}
j & k & l \\
m & n & o \\
p & q & r
\end{bmatrix}
=
\begin{bmatrix}
2 & 19 & 19 \\
9 & 18 & 11 \\
20 & 11 & 22
\end{bmatrix}
\]

Definimos las ecuaciones que se están realizando dentro de esta multiplicación

\begin{align*}
M1_{11}&=1.j + 3.m + 0.p &= 2\\
M1_{12}&=1.k + 3.n + 0.q &= 19\\
M1_{13}&=1.l + 3.o + 0.r &= 19\\
M1_{21}&=5.j + 8.m + 12.p &= 9\\
M1_{22}&=5.k + 8.n + 12.q &= 18\\
M1_{23}&=5.l + 8.o + 12.r &= 11\\
M1_{31}&=9.j + 4.m + 10.p &= 20\\
M1_{32}&=9.k + 4.n + 10.q &= 11\\
M1_{33}&=9.l + 4.o + 10.q &= 22
\end{align*}
\newpage
\subsection{M1: primer sistema de ecuaciones}
Separamos las ecuaciones para $M1_{11}, M1_{21}, M1_{31}$ en un sistema de ecuaciones ya que tienen las mismas variables:
\begin{equation*}
\systeme{
1.j + 3.m = 2,
5.j + 8.m + 12.p = 9 (-10),
9.j + 4.m + 10.p = 20 (12)
}
\end{equation*}
Multiplicamos la 2da y 3ra ecuación
\begin{equation*}
\systeme{-50.j - 80.m - 120.p = -90,
108.j + 48.m + 120.p = 240
}
\end{equation*}
\begin{equation*}
58.j - 32.m = 150
\end{equation*}
Incluímos la ecuación resultante en un sistema de ecuaciones 2x2 junto con la 1ra ecuación del sistema anterior:
\begin{equation*}
\systeme{
58.j - 32.m = 150 (1),
j - 3.m = 2(-58)
}
\end{equation*}
Multiplicamos las ecuaciones y reducimos:
\begin{align*}
58.j - 32.m = 150\\
-58.j - 174.m = -116\\
-206.m = 34\\
m = \frac{-17}{103}
\end{align*}
Reusamos la primera ecuación $j + 3.m = 2$ para despejar $j$
\begin{align*}
j + 3.(\frac{-17}{103}) = 2 \\
j = 2 + \frac{51}{103} = \frac{257}{103}
\end{align*}

Reusamos la segunda ecuación $5.j + 8.m + 12.p$ para despejar $p$
\begin{align*}
5.(\frac{257}{103}) + 8.(\frac{-17}{103}) + 12.p = 9\\
\frac{1285 - 136 - 927}{103} = -12 . p\\
12 . p = \frac{-222}{103}\\
p = \frac{-37}{206}
\end{align*}

\subsection{M1: segundo sistema de ecucaciones}
Separamos las ecuaciones para $M1_{12}, M1_{22}, M1_{32}$ en un sistema de ecuaciones
\begin{equation*}
\systeme{
1.k + 3.n = 19,
5.k + 8.n + 12.q = 18 (-10),
9.k + 4.n + 10.q = 11 (12)
}
\end{equation*}
Multiplicamos las ecuaciones

\begin{equation*}
\systeme{-50.k - 80.n - 120.q = -180,
108.k + 48.n + 120.q = 132
}
\end{equation*}
\begin{equation*}
58.k - 32.n = -48
\end{equation*}
Incluímos la ecuación resultante en un sistema de ecuaciones 2x2 junto con la 1ra ecuación del sistema anterior:
\begin{equation*}
\systeme{
58.k - 32.n = -48 (-1),
k - 3.n = 19 (58)
}
\end{equation*}
Multiplicamos las ecuaciones y reducimos:
\begin{align*}
-58.k + 32.n = 48\\
58.k + 174.n = 1102\\
206.n = 1150\\
n = \frac{575}{103}
\end{align*}
Reusamos la primera ecuación $k + 3.n = 19$ para despejar $k$
\begin{align*}
n + 3.(\frac{575}{103}) = 19 \\
n = 19 - \frac{1725}{103} = \frac{232}{103}
\end{align*}

Reusamos la segunda ecuación $5.k + 8.n + 12.q$ para despejar $q$
\begin{align*}
5.(\frac{232}{103}) + 8.(\frac{575}{103}) + 12.q = 18\\
\frac{1160 + 4600 - 1854}{103} = -12 . q\\
q = \frac{-3906}{1236}\\
q = \frac{-651}{206}
\end{align*}
\pagebreak
\subsection{M1: Tercer sistema de ecuaciones}

Separamos las ecuaciones para $M1_{13}, M1_{23}, M1_{33}$ en un sistema de ecuaciones
\begin{equation*}
\systeme{
1.l + 3.o = 19,
5.l + 8.o + 12.r = 11 (-10),
9.l + 4.o + 10.r = 22 (12)
}
\end{equation*}
Multiplicamos las ecuaciones

\begin{equation*}
\systeme{-50.l - 80.o - 120.r = -110,
108.l + 48.o + 120.r = 264
}
\end{equation*}
\begin{equation*}
58.l - 32.o = 154
\end{equation*}
Incluímos la ecuación resultante en un sistema de ecuaciones 2x2 junto con la 1ra ecuación del sistema anterior:
\begin{equation*}
\systeme{
58.l - 32.o = 154,
l - 3.o = 19 (-58)
}
\end{equation*}
Multiplicamos la ecuación y reducimos:
\begin{align*}
58.l - 32.o = 154\\
-58.l - 174.o = -1102\\
-206.o = -948\\
o = \frac{474}{103}
\end{align*}
Reusamos la primera ecuación $l + 3.o = 19$ para despejar $l$
\begin{align*}
l + 3.(\frac{474}{103}) = 19 \\
l = 19 - \frac{1422}{103} = \frac{535}{103}
\end{align*}

Reusamos la segunda ecuación $5.l + 8.o + 12.r$ para despejar $r$
\begin{align*}
5.(\frac{535}{103}) + 8.(\frac{474}{103}) + 12.r = 11\\
\frac{2675+3792-1133}{103} = -12 . r\\
r = \frac{-5334}{1236}\\
r = \frac{-889}{206}
\end{align*}
\pagebreak
\subsection{M1: Resultado}
Tras estos sistemas de ecuaciones, tenemos
\[
A = \begin{bmatrix}
1 & 3 & 0\\
5 & 8 & 12\\
9 & 4 & 10
\end{bmatrix}, B = \begin{bmatrix}
\frac{257}{103} & \frac{232}{103} & \frac{535}{103} \\
\frac{-17}{103} & \frac{575}{103} & \frac{474}{103} \\
\frac{-37}{206} & \frac{-651}{206} & \frac{-889}{206}
\end{bmatrix}
\]
\pagebreak
\section{Verificacion de claves mediante M1 y M2}
Bit SKV es un algoritmo para claves de autenticación de los productos de Bit SRL, los cuales tendrían como matrices A,C o B,D una tabla de datos que permita identificar al cliente y la otra matriz despejada a partir de la misma. Posteriormente, ambas se expresarían bajo la siguiente estructura de $((3*3*2*2)+(2*2*2*2))*2 = 104$ bytes:

Cada número se representa con 4 bytes, 2 para su numerador (firmado) y 2 para su denominador (no firmado).

Los primeros 36 bytes representan los números de A, donde $R_{i} = A_{(i idiv 3)+1,(i mod 3)+1}$, o sea los números de la misma fila se almacenan conjuntamente. Visualmente, esto parece
\begin{center}

\begin{tabular}{|*{8}{p{0.7cm}|}}
  \hline
  \rowcolor{WhiteSmoke!60!Lavender}$A_{11}$& $A_{12}$ & $A_{13}$ & $A_{21}$ & $A_{22} $& \dots & $A_{32}$ &$A_{33}$\rule{0pt}{0.5cm} \\
  \hline
\end{tabular}
\end{center}

Los próximos 36 bytes representan los números de B del mismo modo. Los últimos 32 bytes representan los números de C y D del mismo modo.

Estos 104 bytes son codificados mediante \textit{base64}, un sistema de codificación binaria que utiliza caracteres del abecedario + dígitos + algunos caracteres especiales para codificar de una manera legible información en formato binario. Por ejemplo, nuestros ABCD de ejemplo en formato b64 serían el string (sin rotura de líneas)
\begin{lstlisting}
AQABAAMAAQAAAAEABQABAAgAAQAMAAEACQABAAQAAQAKAAEAAQFnAOgAZwAXAmcA7/9nAD8C
ZwDaAWcA2//OAHX9zgCH/M4ABQABAAwAAQAIAAEAAAABAAMAAgABAAgAFwAYAJsAYAA=
\end{lstlisting}

\pagebreak
\subsection{Programa ejemplo usando estas matrices}
Un programa ejemplo, que encodea y decodea estas matrices (y muestra el resultado de A*B y C*D) es
\lstset{
	language=[Visual]Basic,
	numbers=left,
	breaklines=true,
}

\begin{lstlisting}
Imports System
Imports System.Linq
Imports Fractions

Public Structure Matrix
    Public Values(,) as Fraction
    Public ReadOnly Property Rows as Integer
    	   Get
		Return Values.GetLength(0)
	   End Get
    End Property

    Public ReadOnly Property Columns as Integer
    	   Get
		Return Values.GetLength(1)
	   End Get
    End Property
    
    Public Sub New(mat as IEnumerable(Of IEnumerable(Of Fraction)))
    	   ReDim Values(mat.Count-1, mat(0).Count-1)
	   For i = 0 to mat.Count-1
	       For j = 0 to mat(0).Count-1
	       	   Values(i,j) = mat(i)(j)
	       Next
	   Next
    End Sub

    Public Overrides Function ToString() As String
    	   Dim str = ""
	   For y = 0 to Columns-1
	       For x = 0 to Rows-1
	       	   str += Values(y,x).ToString()
		   str += vbTab
	       Next
	       str += vbNewLine
	   Next
	   Return str
    End Function

    Public Shared Operator *(ByVal m1 as Matrix, ByVal m2 as Matrix) as Matrix
    	   If m1.Columns <> m2.Rows Then
	      Throw New Exception("m1.Columns must be = m2.Rows")
	   End If
	   Dim values as New List(Of List(Of Fraction))
	   For i = 0 to m1.Rows-1
	       Dim row = new List(Of Fraction)
	       values.Add(row)
	       For j = 0 to m2.Columns-1
	       	   Dim val as Fraction = 0
		   For k = 0 to m2.Rows-1
		       val += (m1.Values(i,k) * m2.Values(k,j))
		   Next
		   row.Add(val)
	       Next
	   Next
	   Return New Matrix(values)
    End Operator
End Structure

Module Program
   Sub Encode()
       Dim input = Console.ReadLine
       Dim values_a = input.Split(";").Select(Function(x) x.Split(",").Select(Function(i) i.Split("/"))).ToList()
       input = Console.ReadLine
       Dim values_b = input.Split(";").Select(Function(x) x.Split(",").Select(Function(i) i.Split("/"))).ToList()
       input = Console.ReadLine
       Dim values_c = input.Split(";").Select(Function(x) x.Split(",").Select(Function(i) i.Split("/"))).ToList()
       input = Console.ReadLine
       Dim values_d = input.Split(";").Select(Function(x) x.Split(",").Select(Function(i) i.Split("/"))).ToList()
       Dim bytevals(103) as Byte
       For x = 0 to 2
       	   For y = 0 to 2
	   Try
	       Dim num_d() as Byte = BitConverter.GetBytes(CType(Integer.Parse(values_a(y)(x)(0)), Int16))
	       bytevals((y*3*4)+(x*4)) = num_d(0)
	       bytevals((y*3*4)+(x*4)+1) = num_d(1)
	       Dim div_d() as Byte = BitConverter.GetBytes(CType(Integer.Parse(values_a(y)(x)(1)), Int16))
	       bytevals((y*3*4)+(x*4)+2) = div_d(0)
	       bytevals((y*3*4)+(x*4)+3) = div_d(1)
	   Catch e as Exception
	       Console.WriteLine(e)
	       Console.WriteLine(x)
	       Console.WriteLine(y)
	   End Try
	   Next
       Next
       For x = 0 to 2
       	   For y = 0 to 2
	       Dim num_d() as Byte = BitConverter.GetBytes(CType(Integer.Parse(values_b(y)(x)(0)), Int16))
	       bytevals((y*4*3)+(x*4)+36) = num_d(0)
	       bytevals((y*4*3)+(x*4)+37) = num_d(1)
	       Dim div_d() as Byte = BitConverter.GetBytes(CType(Integer.Parse(values_b(y)(x)(1)), Int16))
	       bytevals((y*4*3)+(x*4)+38) = div_d(0)
	       bytevals((y*4*3)+(x*4)+39) = div_d(1)
	   Next
       Next
       For x = 0 to 1
       	   For y = 0 to 1
	       Dim num_d() as Byte = BitConverter.GetBytes(CType(Integer.Parse(values_c(y)(x)(0)), Int16))
	       bytevals((y*4*2)+(x*4)+72) = num_d(0)
	       bytevals((y*4*2)+(x*4)+73) = num_d(1)
	       Dim div_d() as Byte = BitConverter.GetBytes(CType(Integer.Parse(values_c(y)(x)(1)), Int16))
	       bytevals((y*4*2)+(x*4)+74) = div_d(0)
	       bytevals((y*4*2)+(x*4)+75) = div_d(1)
	   Next
       Next
       For x = 0 to 1
       	   For y = 0 to 1
	       Dim num_d() as Byte = BitConverter.GetBytes(CType(Integer.Parse(values_d(y)(x)(0)), Int16))
	       bytevals((y*4*2)+(x*4)+88) = num_d(0)
	       bytevals((y*4*2)+(x*4)+89) = num_d(1)
	       Dim div_d() as Byte = BitConverter.GetBytes(CType(Integer.Parse(values_d(y)(x)(1)), Int16))
	       bytevals((y*4*2)+(x*4)+90) = div_d(0)
	       bytevals((y*4*2)+(x*4)+91) = div_d(1)
	   Next
       Next
       Console.WriteLine(System.Convert.ToBase64String(bytevals))
   End Sub
   
   Sub Main(args() as String)
       Dim input = Console.ReadLine
       If input = "ENC" Then
       	  Encode()
	  Return
       End If
       Dim bytes = System.Convert.FromBase64String(input)
       Dim A(2)() as Fraction
       Dim B(2)() as Fraction
       For y = 0 to 2
       	   A(y) = new Fraction(){Nothing, Nothing, Nothing}
           For x = 0 to 2
	       Dim num_a as Int16 = BitConverter.ToInt16(bytes,x*4+y*3*4)
	       Dim den_a as Int16 = BitConverter.ToInt16(bytes,x*4+y*3*4+2)
	       Dim frac = new Fraction(num_a, den_a)
	       A(y)(x) = frac
	   Next
       Next
       For y = 0 to 2
       	   B(y) = new Fraction(){Nothing, Nothing, Nothing}
           For x = 0 to 2
	       Dim num_b as Int16 = BitConverter.ToInt16(bytes,x*4+y*3*4+36)
	       Dim den_b as Int16 = BitConverter.ToInt16(bytes,x*4+y*3*4+38)
	       Dim frac = new Fraction(num_b, den_b)
	       B(y)(x) = frac
	   Next
       Next
       Dim C(1)() as Fraction
       Dim D(1)() as Fraction
       For y = 0 to 1
       	   C(y) = new Fraction(){Nothing, Nothing, Nothing}
           For x = 0 to 1
	       Dim num_a as Int16 = BitConverter.ToInt16(bytes,x*4+y*2*4+72)
	       Dim den_a as Int16 = BitConverter.ToInt16(bytes,x*4+y*2*4+72+2)
	       Dim frac = new Fraction(num_a, den_a)
	       C(y)(x) = frac
	   Next
       Next
       For y = 0 to 1
       	   D(y) = new Fraction(){Nothing, Nothing, Nothing}
           For x = 0 to 1
	       Dim num_b as Int16 = BitConverter.ToInt16(bytes,x*4+y*2*4+88)
	       Dim den_b as Int16 = BitConverter.ToInt16(bytes,x*4+y*2*4+88+2)
	       Dim frac = new Fraction(num_b, den_b)
	       D(y)(x) = frac
	   Next
       Next
       Dim MList as new List(Of List(Of Fraction))
       For y = 0 to 2
       	   Dim nList as New List(Of Fraction)
	   MList.Add(nList)
       	   For x = 0 to 2
	       nList.Add(A(y)(x))
	   Next
       Next
       Dim AMatrix = New Matrix(MList)
       MList = new List(Of List(Of Fraction))
       For y = 0 to 2
       	   Dim nList as New List(Of Fraction)
	   MList.Add(nList)
       	   For x = 0 to 2
	       nList.Add(B(y)(x))
	   Next
       Next
       Dim BMatrix = New Matrix(MList)
       Console.WriteLine("Matrix A:")
       Console.WriteLine(AMatrix)
       Console.WriteLine("Matrix B:")
       Console.WriteLine(BMatrix)
       Dim M1Matrix = AMatrix * BMatrix
       Console.WriteLine("Matrix M1:")
       Console.WriteLine(M1Matrix)
       MList = new List(Of List(Of Fraction))
       For y = 0 to 1
       	   Dim nList as New List(Of Fraction)
	   MList.Add(nList)
       	   For x = 0 to 1
	       nList.Add(C(y)(x))
	   Next
       Next
       Dim CMatrix = New Matrix(MList)
       MList = new List(Of List(Of Fraction))
       For y = 0 to 1
       	   Dim nList as New List(Of Fraction)
	   MList.Add(nList)
       	   For x = 0 to 1
	       nList.Add(D(y)(x))
	   Next
       Next
       Dim DMatrix = New Matrix(MList)
       Dim M2Matrix = CMatrix*DMatrix
       Console.WriteLine("Matrix C:")
       Console.WriteLine(CMatrix)
       Console.WriteLine("Matrix D:")
       Console.WriteLine(DMatrix)
       Console.WriteLine("Matrix M2:")
       Console.WriteLine(M2Matrix)
   End Sub
End Module
\end{lstlisting}
Al ejecutarlo con nuestro string, se da el siguiente resultado
\begin{lstlisting}
>kouta@koutas-lair ~/matrix_mult (git)-[master] % dotnet run
>AQABAAMAAQAAAAEABQABAAgAAQAMAAEACQABAAQAAQAKAAEAAQFnAOgAZwAXAmcA7/9nAD8CZwDaAWcA2//OAHX9zgCH/M4ABQABAAwAAQAIAAEAAAABAAMAAgABAAgAFwAYAJsAYAA=
Matrix A:
1	3	0/0	
5	8	12	
9	4	10	

Matrix B:
257/103	232/103	535/103	
-17/103	575/103	474/103	
-37/206	-651/206	-889/206	

Matrix M1:
2	19	19	
9	18	11	
20	11	22	

Matrix C:
5	12	
8	0/0	

Matrix D:
3/2	1/8	
23/24	155/96	

Matrix M2:
19	20	
12	1	
\end{lstlisting}
\end{document}